
\documentclass[12pt,oneside]{article}
\usepackage{geometry}
\geometry{
    a4paper,
    left=20mm,
    right=20mm,
    top=18.5mm,
    bottom=15mm,
    bindingoffset=0mm,
    footskip=5pt,
}
\usepackage{xeCJK,lmodern}
\setmainfont{Times New Roman}
\setsansfont{Times New Roman}
\setCJKmainfont{simsun.ttc}[Path=./fonts/, AutoFakeSlant, AutoFakeBold={5}]
\setCJKsansfont{simhei.ttf}[Path=./fonts/, AutoFakeSlant, AutoFakeBold={5}]
\setCJKmonofont{Kaiti_GB2312.ttf}[Path=./fonts/, AutoFakeSlant, AutoFakeBold={5}]
\newCJKfontfamily\declarationfont{simfang.ttf}[Path=./fonts/, AutoFakeSlant, AutoFakeBold={5}]
\usepackage{graphicx,float}
\graphicspath{{figures/}}
\usepackage{fancyhdr}
\usepackage{tcolorbox}
\tcbuselibrary{breakable}
\usepackage{caption,subcaption,setspace,titlesec}
%- section
\titleformat{\section}[hang]{\sffamily\bfseries\fontsize{14pt}{20pt}\selectfont}{\arabic{section}}{14pt}{}
\titlespacing{\section}{0pt}{24pt}{6pt}
%- subsection
\titleformat{\subsection}[hang]{\sffamily\bfseries\fontsize{12pt}{20pt}\selectfont}{\arabic{section}.\arabic{subsection}}{12pt}{}
\titlespacing{\subsection}{0pt}{12pt}{6pt}
\titleformat{\subsubsection}[hang]{\sffamily\fontsize{12pt}{20pt}\selectfont}{\arabic{section}.\arabic{subsection}.\arabic{subsubsection}}{12pt}{}
\titlespacing{\subsubsection}{0pt}{12pt}{6pt}
\renewcommand\figurename{图}
\renewcommand\tablename{表}
\DeclareCaptionFont{captionfont}{\rmfamily\fontsize{11pt}{\baselineskip}\selectfont}
\captionsetup{font=captionfont,labelsep=quad,skip=6pt,subrefformat=parens}
\titleformat{\paragraph}[runin]{\rmfamily\fontsize{12pt}{20pt}\selectfont}{}{}{}
\titlespacing{\paragraph}{0pt}{0pt}{0pt}
\setlength{\parskip}{0pt}

\usepackage{amsmath,amssymb,amsfonts,amsthm,bm} % mathamatics
\usepackage{tabularray}
\SetTblrInner{rowsep=5pt}

\usepackage[sort&compress]{gbt7714}
\bibliographystyle{gbt7714-2005-numerical}
\renewcommand\refname{\fontsize{12pt}{20pt}\selectfont\bfseries 参考文献}
\usepackage{hyperref}

\begin{document}
\center{\fontsize{14pt}{20pt}\selectfont\bfseries 表1-1:研究生学位论文开题报告(首次开题)}
\vspace{7pt}

\centering\begin{tblr}{|X[1.2,c]|X[c]|X[c]|}
\hline
拟撰写学位论文的题目 & \SetCell[c=2]{c} & \\
\hline
支持学位论文研究的科研项目 & \SetCell[c=2]{c} &
\end{tblr}

\begin{tcolorbox}[
    colback=white,
    breakable=true,
    sharp corners,
    boxsep=0mm,
    boxrule=0.5pt,
    before skip=0mm,
    after skip=0mm,
    space to=\myspace,
]
开题报告内容\textbf{(博士不少于5000字,硕士不少于3000字)}: 
\setlength{\parindent}{2em}% ====================================================================
% 使用说明
% ====================================================================
% 本文档为开题报告正文模板,请将以下示例内容替换为您的实际开题报告内容。
% 
% 1. 章节标题使用方法:
%    \section{一级标题}
%    \subsection{二级标题}
%    \subsubsection{三级标题}
%
% 2. 参考文献引用方法:
%    使用 \cite{引用标签} 命令,例如:\cite{knuth1984texbook}
%    多个文献引用:\cite{knuth1984texbook,lamport1994latex}
%
% 3. 公式添加方法:
%    行内公式:$E=mc^2$
%    独立公式:
%    \begin{equation}
%        E = mc^2
%    \end{equation}
%
% 4. 列表添加方法:
%    无序列表:
%    \begin{itemize}
%        \item 第一项
%        \item 第二项
%    \end{itemize}
%    
%    有序列表:
%    \begin{enumerate}
%        \item 第一项
%        \item 第二项
%    \end{enumerate}
% ====================================================================

\section{研究背景与意义}

这里是研究背景的正文内容。您可以在这里阐述研究的背景、现状和意义。例如,根据相关研究\cite{wang2019a},该领域存在以下问题……

\subsection{研究背景}

这是一个二级标题示例。可以使用行内公式,如爱因斯坦质能方程 $E=mc^2$,或者使用独立公式:

\begin{equation}
    \int_{a}^{b} f(x) \, dx = F(b) - F(a)
    \label{eq:fundamental}
\end{equation}

其中,公式\ref{eq:fundamental}表示微积分基本定理。

\subsection{研究意义}

研究意义可以分为以下几点:

\begin{itemize}
    \item 理论意义:丰富相关理论体系
    \item 实践意义:为实际应用提供指导
    \item 创新意义:提出新的研究方法
\end{itemize}

\section{文献综述}

文献综述部分需要对现有研究进行梳理和评述\cite{author2023,coauthor2022}。

\subsection{国内研究现状}

有序列表示例:

\begin{enumerate}
    \item 第一个研究方向的现状分析
    \item 第二个研究方向的现状分析
    \item 第三个研究方向的现状分析
\end{enumerate}

\subsection{国外研究现状}

国外学者在该领域也进行了大量研究,取得了显著成果。

\section{研究内容与方法}

\subsection{研究内容}

本研究的主要内容包括:

\begin{itemize}
    \item 内容一:具体描述
    \item 内容二:具体描述
    \item 内容三:具体描述
\end{itemize}

\subsection{研究方法}

采用的研究方法如下:

\begin{enumerate}
    \item 文献研究法
    \item 实验研究法
    \item 数据分析法
\end{enumerate}

\section{研究计划与预期成果}

\subsection{研究计划}

详细的研究时间安排和阶段性目标。

\begin{figure}[H]
\centering
\includegraphics[width=0.8\linewidth]{1.png} 
\caption{研究计划时间表}
\end{figure}

\subsection{预期成果}

预期将取得以下成果,并发表相关论文\cite{future2024}。
% ====================================================================
% 注意:请在 references.bib 文件中添加对应的参考文献条目
% ==================================================================== 
\bibliography{references}
\end{tcolorbox}

\centering\begin{tblr}{|X[2,l]|X[c]|}
\SetCell[c=2]{c} {\bfseries 开题报告指导老师审阅意见} & \\
\hline
对开题报告进行审阅,是否同意进入专家评审环节 & $\Box$ 是 \qquad $\Box$ 否 \\
\hline
\SetCell[c=2]{l} 请填写审阅意见:& \\
[48pt]
\SetCell[c=2]{halign=r,valign=b} 
指导老师签名:\hspace{42pt}

年 \quad 月 \quad 日
& \\
\hline
\end{tblr}

\end{document}


