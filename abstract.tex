% TODO: 设置摘要
\begin{abstract}
    \LaTeX 是一种强大的排版系统,该系统提供了丰富的功能,包括自动编号、交叉引用、数学公式排版、参考文献管理等。它还支持多种文档类、宏包和模板,使用户能够根据自己的需求定制文档的外观和格式,广泛用于撰写学术论文、科技文档和出版物。
    本文主要以前期准备、图表设置、使用Zotero管理参考文献、\LaTeX 部分所需包介绍、附录和参考格式等环节进行介绍。
    首先在前期准备中介绍了Git的相关内容以及VS Code的配置。
    此外,还介绍了文件和标签的命名规则,为编写论文的命名提供参考,并对在\LaTeX 中的图表设置以例子详细说明如何在论文中插入和设置图表,涵盖了图片的插入、标注、引用和交叉引用等方面的技巧和方法,   
    也对如何使用Zotero管理参考文献进行了介绍说明。
    最后,在附录部分给出了参考格式的理解说明。    
    通过阅读本文,读者将获得关于前期准备、图表设置、参考文献管理、\LaTeX 必需包、附录和参考格式的全面指导,有助于顺利完成论文写作和格式规范要求。

\end{abstract}
\keywords{\LaTeX;模板学习}


% NOTE:以下是摘要的英文部分,需要时可取消注释
% \begin{abstractEn}
% fdsa
% \end{abstractEn}
% \keywordsEn{keyword1; keyword2; keyword3; keyword4; keyword5; keyword6; keyword7; keyword8}
